%% LyX 1.3 created this file.  For more info, see http://www.lyx.org/.
%% Do not edit unless you really know what you are doing.
\documentclass[english, 12pt]{article}
\usepackage{times}
%\usepackage{algorithm2e}
\usepackage{url}
\usepackage{bbm}
\usepackage[T1]{fontenc}
\usepackage[latin1]{inputenc}
\usepackage{geometry}
\geometry{verbose,letterpaper,tmargin=2cm,bmargin=2cm,lmargin=1.5cm,rmargin=1.5cm}
\usepackage{rotating}
\usepackage{color}
\usepackage{graphicx}
\usepackage{subcaption}
\usepackage{amsmath, amsthm, amssymb}
\usepackage{setspace}
\usepackage{lineno}
\usepackage{hyperref}
\usepackage{bbm}
\usepackage{makecell}

%\renewcommand{\arraystretch}{1.8}

\usepackage{xr}
\externaldocument{paper-ldpred2-supp}

%\linenumbers
%\doublespacing
\onehalfspacing
%\usepackage[authoryear]{natbib}
\usepackage{natbib} \bibpunct{(}{)}{;}{author-year}{}{,}

%Pour les rajouts
\usepackage{color}
\definecolor{trustcolor}{rgb}{0,0,1}

\usepackage{dsfont}
\usepackage[warn]{textcomp}
\usepackage{adjustbox}
\usepackage{multirow}
\usepackage{graphicx}
\graphicspath{{figures/}}
\DeclareMathOperator*{\argmin}{\arg\!\min}

\let\tabbeg\tabular
\let\tabend\endtabular
\renewenvironment{tabular}{\begin{adjustbox}{max width=0.9\textwidth}\tabbeg}{\tabend\end{adjustbox}}

\makeatletter

%%%%%%%%%%%%%%%%%%%%%%%%%%%%%% LyX specific LaTeX commands.
%% Bold symbol macro for standard LaTeX users
%\newcommand{\boldsymbol}[1]{\mbox{\boldmath $#1$}}

%% Because html converters don't know tabularnewline
\providecommand{\tabularnewline}{\\}

\usepackage{babel}
\makeatother


\begin{document}


\title{LDpred2: better, faster, stronger}
\author{Florian Priv\'e,$^{\text{1,}*}$ Julyan Arbel,$^{\text{2}}$ and Bjarni J. Vilhj\'almsson$^{\text{1,}*}$}

\date{~ }
\maketitle

\noindent$^{\text{\sf 1}}$National Centre for Register-Based Research, Aarhus University, Aarhus, 8210, Denmark. \\
\noindent$^{\text{\sf 2}}$Univ.\ Grenoble Alpes, Inria, CNRS, Grenoble INP, LJK, Grenoble, 38000, France. \\
\noindent$^\ast$To whom correspondence should be addressed.\\

\noindent Contacts:
\begin{itemize}
\item \url{florian.prive.21@gmail.com}
\item \url{bjv@econ.au.dk}
\end{itemize}

\vspace*{4em}

\abstract{	

}


%%%%%%%%%%%%%%%%%%%%%%%%%%%%%%%%%%%%%%%%%%%%%%%%%%%%%%%%%%%%%%%%%%%%%%%%%%%%%%%%

\clearpage

\section{Introduction}

The use of polygenic scores (PGS) is widespread now.
There are high hopes for using these scores, which sum up the information of many genetic variants into a single score, into clinical practice. 
Predictive performance of PGS is inexorably bounded by heritability, the proportion of phenotypic variance that is attributable to genetics, so that their possibility of being used as a diagnostic tool for diseases is limited.
Nevertheless, it should be possible to use them to stratify the population into groups of different risk levels, to be able to identify individuals at high risk for a given disease [CITATION]. 
These people should then be monitored more closely to detect the disease and intervene sooner to maximize chances of remission (e.g.\ in the case of cancer).
PGS are also extensively used in epidemiology and economics as predictive variables of interest. For example, a recently derived PGS for education attainment has been the most predictive variable in social sciences so far [CITATION].

LDpred is a popular and powerful method for deriving polygenic scores based on summary statistics and a Linkage Disequilibrium (LD) matrix only \cite[]{vilhjalmsson2015modeling}.
It assumes there is a proportion $p$ of variants that are causal.
However, LDpred currently has several limitations that may result in limited predictive performance.
The non-infinitesimal version of LDpred, a Gibbs sampler, is particularly sensitive to model misspecification when applied to summary statistics with large sample sizes.
It may also be very unstable in long-range LD regions such as the human leukocyte antigen (HLA) region of chromosome 6. This issue has lead to the removal of this region from analyses  \cite[]{marquez2018modeling,lloyd2019improved}, which is a shame since this region of the genome contains several disease-associated variants, particularly with autoimmune diseases and psychiatric disorders [CITATIONS].
LDpred is also considered not particularly fast, nor memory-efficient, nor it provides truly sparse models as in its model specification.

Here, we present LDpred2, a new version of LDpred, that aims to cope with the issues presented before.
We provide a faster and more robust implementation of LDpred in R package bigsnpr \cite[]{prive2017efficient}.
We also provide two new options: a ``sparse'' option, where LDpred2 truly fit some effects to be zero, therefore providing a sparse vector of effects; and an ``auto'' option, where LDpred2 automatically estimates the sparsity $p$ and the (SNP) heritability $h^2$, therefore for which there is no need for validation data to choose hyper-parameters performing best.
We show that [DESCRIBE RESULTS].


%%%%%%%%%%%%%%%%%%%%%%%%%%%%%%%%%%%%%%%%%%%%%%%%%%%%%%%%%%%%%%%%%%%%%%%%%%%%%%%%

\section{Results}

\subsection*{Overview of Methods}


%%%%%%%%%%%%%%%%%%%%%%%%%%%%%%%%%%%%%%%%%%%%%%%%%%%%%%%%%%%%%%%%%%%%%%%%%%%%%%%%

\section{Discussion}



%%%%%%%%%%%%%%%%%%%%%%%%%%%%%%%%%%%%%%%%%%%%%%%%%%%%%%%%%%%%%%%%%%%%%%%%%%%%%%%%

\section{Online Methods}

\subsection*{Overview of LDpred model}

LDpred assumes the following model for effect sizes,
\begin{equation}\label{eq:model}
\beta_j \sim \left\{
\begin{array}{ll}
\mathcal N\left(0, \dfrac{h^2}{M p}\right) & \mbox{with probably $p$,} \\
0 & \mbox{otherwise,}
\end{array}
\right.
\end{equation}
where $p$ is the proportion of causal variants, $M$ the number of variants and $h^2$ the (SNP) heritability.
\cite{vilhjalmsson2015modeling} estimates $h^2$ using constrained LD score regression (intercept fixed to 1) and recommend testing a grid of hyper-parameter values for $p$ (e.g.\ 1, 0.3, 0.1, 0.03, 0.01, etc.).

\newcommand{\hmpn}{\dfrac{h^2}{M p} + \dfrac{1}{N}}
\newcommand{\phmpn}{\dfrac{p}{\sqrt{\hmpn}}}
\newcommand{\betahmpn}{\dfrac{\tilde{\beta}_j^2}{\hmpn}}

To estimate effect sizes $\beta_j$, \cite{vilhjalmsson2015modeling} use a Gibbs sampler [REALLY A GS?]. They compute the probability that variant $j$ is causal as 
\[\bar{p}_j = P\left(\beta_j \sim \mathcal N(\cdot,\cdot) ~|~ \tilde{\beta}_j\right) = \dfrac{\phmpn \exp\left\lbrace-\dfrac{1}{2} \betahmpn\right\rbrace}{\phmpn \exp\left\lbrace-\dfrac{1}{2} \betahmpn\right\rbrace + \dfrac{1-p}{\sqrt{\dfrac{1}{N}}} \exp\left\lbrace-\dfrac{1}{2} \dfrac{\tilde{\beta}_j^2}{\dfrac{1}{N}}\right\rbrace}~,\]
which we rewrite as
\begin{equation}
\bar{p}_j = \dfrac{1}{1 + \dfrac{1-p}{p} \sqrt{1 + \dfrac{N h^2}{Mp}} \exp\left\lbrace-\dfrac{1}{2} \dfrac{N \tilde{\beta}_j^2}{1 + \dfrac{Mp}{N h^2}}\right\rbrace}~.
\end{equation}

Then, $\beta_j$ is sampled according to
\begin{equation}\label{eg:beta_random}
\beta_j ~|~ \tilde{\beta}_j \sim \left\{
\begin{array}{ll}
\mathcal N\left(\dfrac{1}{1 + \dfrac{M p}{N h^2}} \tilde{\beta}_j, \dfrac{1}{1 + \dfrac{M p}{N h^2}}\dfrac{1}{N}\right) & \mbox{with probably $\bar{p}_j$,} \\
0 & \mbox{otherwise.}
\end{array}
\right.
\end{equation}

[TO COMPLETE + CLEARER -> SHOULD SHOW THAT IT IS ITERATIVE]

\subsection*{New LDpred2 models}

LDpred2 comes with two extensions of the LDpred model.

The first extension consists in estimating $p$ and $h^2$ in the model, as opposed to testing several values of $p$ and estimating $h2$ using constrained LD score regression \cite[]{bulik2015ld}. This makes LDpred2-auto a method free of hyper-parameters which can therefore directly be applied to data without the need of a validation dataset to choose best-performing hyper-parameters.
To estimate $p$, we count the number of causal variants (i.e.\ $S = \sum_j(\beta_j \neq 0)$ in equation \ref{eg:beta_random}). 
We can assume that $S \sim B(M, p)$, so if we place a prior $p \sim \text{Beta}(1, 1) \equiv \mathcal{U}(0, 1)$, we can sample $p$ from the conjugate prior $p \sim \text{Beta}(1 + S, 1 + M - S)$.
Due to complexity reasons, we could not derive a Bayesian estimator of $h^2$. Instead, we estimate $h^2 = \beta^T K \beta$, where $K$ is the correlation matrix.

The second extension, LDpred2-sparse, aims at providing sparse effect size estimates, i.e. that some resulting effects are exactly 0.
When the sparse solution is sought and when $\bar{p}_j < p$, we set $\beta_j = 0$.
Note that LDpred2-auto does not have a sparse option, but it is possible to run LDpred2-sparse with the $p$ and $h^2$ estimates from LDpred2-auto.

\subsection*{New strategy for local correlation}

LDpred has a window size parameter that needs to be set; for a given variant, correlations with other variants outside of this window are assumed to be 0.
The recommended value for this window (in number of variants) is to use the total number of variants divided by 3000, which corresponds to a window radius of around 2 Mb \cite[]{vilhjalmsson2015modeling}.
After investigating, we come to the conclusion that this window size is not large enough. Indeed, the human leukocyte antigen (HLA) region of chromosome 6 is 8 Mb long \cite[]{price2008long}. 
Using a window of 8Mb would be computationally and memory inefficient.
Instead, we propose to use genetic distances. Genome-wide, 1 Mb corresponds on average to 1 cM. Yet, the HLA region is only 3 cM long (vs.\ 8 Mb long).
Therefore, genetic distances enable to capture the same LD at a smaller distance.
We provide function \texttt{snp\_asGeneticPos()} in package bignspr to easily interpolate physical position to genetic positions.
We recommend to use genetic positions and to use a size parameter of 3 cM when computing the correlation between variants for LDpred2. 
Note that, in the code, we use \texttt{size = 3 / 1000} as parameter \texttt{size} is internally multiplied by 1000 in functions of package bigsnpr.

\subsection*{New strategy for other hyper-parameters}



%%%%%%%%%%%%%%%%%%%%%%%%%%%%%%%%%%%%%%%%%%%%%%%%%%%%%%%%%%%%%%%%%%%%%%%%%%%%%%%%

\clearpage
%\vspace*{5em}

\section*{Software and code availability}

%All code used for this paper is available at \url{https://github.com/privefl/paper4-bedpca/tree/master/code}.
%R packages bigsnpr and bigutilsr can be installed from either CRAN or GitHub (see \url{https://github.com/privefl/bigsnpr}).
%A tutorial on the steps to perform PCA on 1000G data is available at \url{https://privefl.github.io/bigsnpr/articles/bedpca.html}. 

\section*{Acknowledgements}

This research has been conducted using the UK Biobank Resource under Application Number 41181.

F.P. and B.V.\ are supported by the Danish National Research Foundation (Niels Bohr Professorship to Prof. John McGrath), and also acknowledge the Lundbeck Foundation Initiative for Integrative Psychiatric Research, iPSYCH (R248-2017-2003).

\section*{Declaration of Interests}

The authors declare no competing interests.

%%%%%%%%%%%%%%%%%%%%%%%%%%%%%%%%%%%%%%%%%%%%%%%%%%%%%%%%%%%%%%%%%%%%%%%%%%%%%%%%

\clearpage

\bibliographystyle{natbib}
\bibliography{refs}

%%%%%%%%%%%%%%%%%%%%%%%%%%%%%%%%%%%%%%%%%%%%%%%%%%%%%%%%%%%%%%%%%%%%%%%%%%%%%%%%

\end{document}
